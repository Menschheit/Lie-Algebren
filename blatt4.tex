\documentclass[12pt,leqno,twoside]{article} 
\usepackage{leftidx}
\usepackage{bezier,amsmath,amssymb,stmaryrd,lscape,amsthm,upgreek}
\usepackage{bbold}
\usepackage{graphicx}
\usepackage{yfonts}
\graphicspath{ {d:/} }
\usepackage[colorlinks, linkcolor=black, citecolor=black, urlcolor=black]{hyperref}                % hyperlinked pdf-output


\pagestyle{myheadings}
\textheight24.5cm
\textwidth17cm
\unitlength0.1mm        
\oddsidemargin-0.5cm                    % nur in Bielefeld
\evensidemargin-0.5cm                   % nur in Bielefeld
\topmargin-1.0cm                        %     - " -
\voffset=-1.0cm                         % fuer arxiv
\parskip1.2ex
\parindent0pt
\renewcommand{\arraystretch}{1.2}   
\setcounter{secnumdepth}{5}
\setcounter{tocdepth}{5}
\sloppy

% new title sizes
\renewcommand{\contentsname}{\large Contents}
%\renewcommand{\bibname}{\large References}

% macros
\newcommand{\scm}{\scriptstyle}
\newcommand{\U}{\mathcal U}
\newcommand{\lam}{\lambda}
\newcommand{\eps}{\varepsilon}
\newcommand{\del}{\Delta}
\newcommand{\R}{\mathbb R}
\newcommand{\N}{\mathbb N}
\newcommand{\g}{\mathfrak g}
\newcommand{\A}{\mathcal A}
\newcommand{\T}{\mathcal T}
\newcommand{\F}{\mathcal F}
\newcommand{\hsp}[1]{\hspace*{#1mm}}
\newcommand{\vsp}[1]{\vspace*{#1mm}}
\newcommand{\0}{\circ} 
\newcommand{\enger}{\setlength{\arraycolsep}{1pt}
	\renewcommand{\arraystretch}{0.5} }
\newcommand{\weiter}{\setlength{\arraycolsep}{3pt}
	\renewcommand{\arraystretch}{1.0} }
\newcommand{\smatdd}[9]{\enger
	\left(
	\begin{array}{ccc}
		\scm #1 & \scm #2 & \scm #3 \\
		\scm #4 & \scm #5 & \scm #6 \\
		\scm #7 & \scm #8 & \scm #9 \\
	\end{array}
	\right) 
	\weiter }
\newcommand{\smatde}[3]{\enger
	\left(
	\begin{array}{c}
		\scm #1 \\
		\scm #2 \\
		\scm #3
	\end{array}
	\right) 
	\weiter }
\newcommand{\smatzz}[4]{\enger
	\left(
	\begin{array}{cc}
		\scm #1 & \scm #2 \\
		\scm #3 & \scm #4 
	\end{array}
	\right) 
	\weiter }
\newcommand{\lrxa}[1]{\xrightarrow{#1}}
\newcommand{\ba}{\begin{array}}
\newcommand{\barcl}{\begin{array}{rcl}}
\newcommand{\ea}{\end{array}}
\newcommand{\set}[2]{\{\,#1\,:\,#2\,\}}
\newcommand{\auf}{\stackrel}
\DeclareMathOperator{\Char}{char}
\DeclareMathOperator{\ad}{ad}
\newcommand{\mf}{\mathfrak}
\newtheorem{AG}{Aufgabe}
\setcounter{AG}{13}
\begin{document}

\title{Blatt4}
\begin{AG}
\rm
Wir haben eine Abbildung $\ad':  \mf{g} \to \mf{gl}(\mf{a}),\, g\mapsto (\ad_{\mf{g}}g)|_{\mf{a}}^\mf{a}$. Ist wohldefiniert da $(\ad_{\mf{g}}g)(a) = [g,a] \in \mf{a}$.

Wir wissen dass die Abbildung $\ad_{\mf{g}}$ ein Morphismus ist, also ist $\ad'$ auch linear. Weiter gilt f\"ur $g,g' \in \mf{g}$

\[
\barcl
[\ad'g,\, ad'g'] & = & \ad'g \0 \ad'g' - \ad'g'\0\ad'g\\
				 & = &a\mapsto ( [g,\,[g',\,a]] - [g',\,[g,\,a]]) \\
				 & = & a\mapsto [[g,\,g'],a] \\
				 & = &\ad'[g,\,g'] \in \ad'\mf{g}.
\ea
\]

Dann ist $\ad'\mf{g} \leqslant \mf{gl}(\mf{a})$. 


Da $\mf{g}$ nilpotent ist, besteht $\ad_{\mf{g}}\mf{g}$ (nach Engel) aus nilpotenten Endomorphismen von $\mf{g}$. Das hei{\ss}t f\"ur $g\in \mf{g}$ ist $\ad_{\mf{g}}g$ nilpotent, also auch $(\ad_{\mf{g}}g)|_{\mf{a}}^{\mf{a}}$. Daraus folgt dass $\ad'\mf{g}$ aus nilpotenten Endomorphismen besteht.

Dann sind die Voraussetzungen f\"ur Lemma 37 erf\"ullt. Nach dem Lemma gibt es ein $v \in \mf{a}\backslash\{0\}$ mit $[g,v]=(\ad'g)(v) = 0$ f\"ur alle $g \in \mf{g}$, d.h $v \in \mf{z}(\mf{g})$, somit ist $0\neq v \in \mf{a}\cap \mf{z}(\mf{g})$ also $\mf{a}\cap \mf{z}(\mf{g}) \neq 0$.


 
\end{AG}

\begin{AG}
\rm
Wir haben Projektionen
\[
\barcl
\pi_1: \mf{g} \oplus \mf{h} & \to     & \mf{g}\\
					 (g,\,h)&\mapsto  & g
\ea
\]
und
\[
\barcl
\pi_2: \mf{g} \oplus \mf{h} & \to     & \mf{h}\\
(g,\,h)&\mapsto  & h.
\ea
\]

Die sind surjektiven Morphismen von Liealgebren.

Sei $\mf{a}\trianglelefteq \mf{g} \oplus \mf{h}$ mit $\mf{a}$ abelsch, dann ist $\pi_i(\mf{a})$ abelsch. F\"ur $g\in\mf{g}$ und $(a,\, b)\in \mf{a}$ gilt
\[
[\pi_1((a,\,b)),\, g] = [\pi_1((a,\,b)),\, \pi_1((g,\,0))] = \pi_1([(a,\,b),\,(g,\,0)]) \in \pi_1(\mf{a})
\]
Also ist $\pi_1(\mf{a}) \trianglelefteq \mf{g}$, nach Voraussetzung ist dann $\pi_1(\mf{a}) = 0$.

Analog gilt auch  $\pi_2(\mf{a}) = 0$, also ist $\mf{a} = 0$, somit ist $\mf{g} \oplus \mf{h}$ halbeinfach.
\end{AG}

\end{document}