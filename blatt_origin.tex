\documentclass[12pt,leqno,twoside]{book} 
\usepackage{leftidx}
\usepackage{bezier,amsmath,amssymb,stmaryrd,lscape,amsthm,upgreek}
\usepackage{bbold}
\usepackage{graphicx}
\usepackage{yfonts}
\graphicspath{ {d:/} }
\usepackage[colorlinks, linkcolor=black, citecolor=black, urlcolor=black]{hyperref}                % hyperlinked pdf-output


\pagestyle{myheadings}
\textheight24.5cm
\textwidth17cm
\unitlength0.1mm        
\oddsidemargin-0.5cm                    % nur in Bielefeld
\evensidemargin-0.5cm                   % nur in Bielefeld
\topmargin-1.0cm                        %     - " -
\voffset=-1.0cm                         % fuer arxiv
\parskip1.2ex
\parindent0pt
\renewcommand{\arraystretch}{1.2}   
\setcounter{secnumdepth}{5}
\setcounter{tocdepth}{5}
\sloppy

% new title sizes
\renewcommand{\contentsname}{\large Contents}
\renewcommand{\bibname}{\large References}

% macros
\newcommand{\scm}{\scriptstyle}
\newcommand{\U}{\mathcal U}
\newcommand{\lam}{\lambda}
\newcommand{\eps}{\varepsilon}
\newcommand{\del}{\Delta}
\newcommand{\R}{\mathbb R}
\newcommand{\N}{\mathbb N}
\newcommand{\g}{\mathfrak g}
\newcommand{\A}{\mathcal A}
\newcommand{\T}{\mathcal T}
\newcommand{\F}{\mathcal F}
\newcommand{\hsp}[1]{\hspace*{#1mm}}
\newcommand{\vsp}[1]{\vspace*{#1mm}}
\newcommand{\enger}{\setlength{\arraycolsep}{1pt}
	\renewcommand{\arraystretch}{0.5} }
\newcommand{\weiter}{\setlength{\arraycolsep}{3pt}
	\renewcommand{\arraystretch}{1.0} }
\newcommand{\smatdd}[9]{\enger
	\left(
	\begin{array}{ccc}
		\scm #1 & \scm #2 & \scm #3 \\
		\scm #4 & \scm #5 & \scm #6 \\
		\scm #7 & \scm #8 & \scm #9 \\
	\end{array}
	\right) 
	\weiter }
\newcommand{\smatde}[3]{\enger
	\left(
	\begin{array}{c}
		\scm #1 \\
		\scm #2 \\
		\scm #3
	\end{array}
	\right) 
	\weiter }
\newcommand{\lrxa}[1]{\xrightarrow{#1}}
\newcommand{\ba}{\begin{array}}
\newcommand{\barcl}{\begin{array}{rcl}}
\newcommand{\ea}{\end{array}}
\newcommand{\auf}{\stackrel}
\begin{document}\documentclass[12pt,leqno,twoside]{book} 
\usepackage{leftidx}
\usepackage{bezier,amsmath,amssymb,stmaryrd,lscape,amsthm,upgreek}
\usepackage{bbold}
\usepackage{graphicx}
\usepackage{yfonts}
\graphicspath{ {d:/} }
\usepackage[colorlinks, linkcolor=black, citecolor=black, urlcolor=black]{hyperref}                % hyperlinked pdf-output


\pagestyle{myheadings}
\textheight24.5cm
\textwidth17cm
\unitlength0.1mm        
\oddsidemargin-0.5cm                    % nur in Bielefeld
\evensidemargin-0.5cm                   % nur in Bielefeld
\topmargin-1.0cm                        %     - " -
\voffset=-1.0cm                         % fuer arxiv
\parskip1.2ex
\parindent0pt
\renewcommand{\arraystretch}{1.2}   
\setcounter{secnumdepth}{5}
\setcounter{tocdepth}{5}
\sloppy

% new title sizes
\renewcommand{\contentsname}{\large Contents}
\renewcommand{\bibname}{\large References}

% macros
\newcommand{\scm}{\scriptstyle}
\newcommand{\U}{\mathcal U}
\newcommand{\lam}{\lambda}
\newcommand{\eps}{\varepsilon}
\newcommand{\del}{\Delta}
\newcommand{\R}{\mathbb R}
\newcommand{\N}{\mathbb N}
\newcommand{\g}{\mathfrak g}
\newcommand{\A}{\mathcal A}
\newcommand{\T}{\mathcal T}
\newcommand{\F}{\mathcal F}
\newcommand{\hsp}[1]{\hspace*{#1mm}}
\newcommand{\vsp}[1]{\vspace*{#1mm}}
\newcommand{\enger}{\setlength{\arraycolsep}{1pt}
	\renewcommand{\arraystretch}{0.5} }
\newcommand{\weiter}{\setlength{\arraycolsep}{3pt}
	\renewcommand{\arraystretch}{1.0} }
\newcommand{\smatdd}[9]{\enger
	\left(
	\begin{array}{ccc}
		\scm #1 & \scm #2 & \scm #3 \\
		\scm #4 & \scm #5 & \scm #6 \\
		\scm #7 & \scm #8 & \scm #9 \\
	\end{array}
	\right) 
	\weiter }
\newcommand{\smatde}[3]{\enger
	\left(
	\begin{array}{c}
		\scm #1 \\
		\scm #2 \\
		\scm #3
	\end{array}
	\right) 
	\weiter }
\newcommand{\smatzz}[4]{\enger
	\left(
	\begin{array}{cc}
		\scm #1 & \scm #2 \\
		\scm #3 & \scm #4 
	\end{array}
	\right) 
	\weiter }
\newcommand{\lrxa}[1]{\xrightarrow{#1}}
\newcommand{\ba}{\begin{array}}
\newcommand{\barcl}{\begin{array}{rcl}}
\newcommand{\ea}{\end{array}}
\newcommand{\auf}{\stackrel}
\newtheorem{AG}{Aufgabe}

\begin{document}
\begin{AG}
\rm
(1). Sei $\g=K^{3\times1}$ mit $\left[\left(\begin{array}{c} \alpha \\ \beta \\ \gamma \end{array}\right),\left(\begin{array}{c} \alpha' \\ \beta' \\ \gamma' \end{array}\right)\right]=\left(\begin{array}{c} \beta\gamma'-\gamma\beta' \\ \gamma\alpha'-\alpha\gamma' \\ \alpha\beta'-\beta\alpha' \end{array}\right)$ f\"{u}r $\left(\begin{array}{c} \alpha \\ \beta \\ \gamma \end{array}\right),\left(\begin{array}{c} \alpha' \\ \beta' \\ \gamma' \end{array}\right)\in K^{3\times1}$, 
\\Zu zeigen : $\g$ ist eine Lie-Algebra.

Die Abbildung ist bilinear.

1.F\"{u}r $g=\left(\begin{array}{c} \alpha \\ \beta \\ \gamma \end{array}\right)\in K^{3\times1}$, $[g,g]=\left[\left(\begin{array}{c} \alpha \\ \beta \\ \gamma \end{array}\right),\left(\begin{array}{c} \alpha \\ \beta \\ \gamma \end{array}\right)\right]=\left(\begin{array}{c} \beta\gamma-\gamma\beta \\ \gamma\alpha-\alpha\gamma \\ \alpha\beta-\beta\alpha \end{array}\right)=0$\\
2.Seien $g,g',g''\in K^{3\times1}$ mit $g=\left(\begin{array}{c} \alpha \\ \beta \\ \gamma \end{array}\right),g'=\left(\begin{array}{c} \alpha' \\ \beta' \\ \gamma' \end{array}\right),g''=\left(\begin{array}{c} \alpha'' \\ \beta'' \\ \gamma'' \end{array}\right)$\\
Dann gilt 
\[[g,[g',g'']]+[g',[g'',g']]+[g'',[g,g']]=\left(\begin{array}{c} \beta\alpha'\beta''-\beta\beta'\alpha''-\gamma\gamma'\alpha''+\gamma\alpha'\gamma'' \\ \gamma\beta'\gamma''-\gamma\gamma'\beta''-\alpha\alpha'\beta''+\alpha\beta'\alpha'' \\ \alpha\gamma'\alpha''-\alpha\alpha'\gamma''-\beta\beta'\gamma''+\beta\gamma'\beta'' \end{array}\right)+\]\[\left(\begin{array}{c} \beta'\alpha''\beta-\beta'\beta''\alpha-\gamma'\gamma''\alpha+\gamma'\alpha''\gamma \\ \gamma'\beta''\gamma-\gamma'\gamma''\beta-\alpha'\alpha''\beta+\alpha'\beta''\alpha \\ \alpha'\gamma''\alpha-\alpha'\alpha''\gamma-\beta'\beta''\gamma+\beta'\gamma''\beta \end{array}\right)+\left(\begin{array}{c} \beta''\alpha\beta'-\beta''\beta\alpha'-\gamma''\gamma\alpha'+\gamma''\alpha\gamma' \\ \gamma''\beta\gamma'-\gamma''\gamma\beta'-\alpha''\alpha\beta'+\alpha''\beta\alpha' \\ \alpha''\gamma\alpha'-\alpha''\alpha\gamma'-\beta''\beta\gamma'+\beta''\gamma\beta' \end{array}\right)=0\]
Dann ist $\g$ eine Lie-Algebra.\\
\\
(2). Sei $n\geq0, a\in K^{n\times n}$, zu zeigen: $\mathfrak{o}(K,a):=\{g\in\mathfrak{gl}_n(K):g^ta=-ag\}\leq\mathfrak{gl}_n(k)$\\
Es ist zu zeigen: Seien $g,\bar{g}\in\mathfrak{o}(K,a)$, $[g,\bar{g}]\in\mathfrak{o}(K,a)\Leftrightarrow[g,\bar{g}]^ta=-a[g,\bar{g}]$\\
\[[g,\bar{g}]^ta=(g\bar{g}-\bar{g}g)^ta=(g\bar{g})^ta-(\bar{g}g)^ta=\bar{g}^tg^ta-g^t\bar{g}^ta\]
Da $g,\bar{g}\in\mathfrak{o}(K,a)\Rightarrow g^ta=-ag,\bar{g}^ta=-a\bar{g}$\\
\[\bar{g}^tg^ta-g^t\bar{g}^ta=-\bar{g}^tag+g^ta\bar{g}=a\bar{g}g-ag\bar{g}=-a(g\bar{g}-\bar{g}g)=-a[g,\bar{g}]\]
$[g,\bar{g}]^ta=-a[g,\bar{g}]$ also $[g,\bar{g}]\in\mathfrak{o}(K,a)$, dann ist $\mathfrak{o}(K,a)\leq\mathfrak{gl}_n(k)$\\
\\
(3). Sei $f:\mathfrak{h}\rightarrow\mathfrak{k}$ ein Isomorphismus von Liealgebren, zu zeigen: $f^{-1}:\mathfrak{k}\rightarrow\mathfrak{h}$ ist auch ein Isomorphismus.\\
Es ist zu zeigen : f\"{u}r $k,k'\in\mathfrak{k}, f^{-1}([k,k'])=[f^{-1}(k),f^{-1}(k')]$\\
Da f ist ein Isomorphismus, $[f(f^{-1}(k)),f(f^{-1}(k'))]=f([f^{-1}(k),f^{-1}(k')]$
\[\Rightarrow f^{-1}([k,k'])=f^{-1}([f(f^{-1}(k)),f(f^{-1}(k'))])=f^{-1}\circ f([f^{-1}(k),f^{-1}(k')])=[f^{-1}(k),f^{-1}(k')]\]
Dann ist $f^{-1}$ auch ein Isomorphismus.

(4).Zu zeigen ist: $\g \simeq\mathfrak{o}(K,E_3)$

Wir wissen 
\[
\mathfrak{o}(K,E_3) = \{a_1\cdot u_1 + a_2\cdot u_2 + a_3\cdot u_3 \;|\; a_1,a_2,a_3 \in K\},
\]
wobei $u_1 = \smatdd{0}{1}{0}{-1}{0}{0}{0}{0}{0}$, $u_2=\smatdd{0}{0}{1}{0}{0}{0}{-1}{0}{0}, u_3 = \smatdd{0}{0}{0}{0}{0}{1}{0}{-1}{0}$, die bilden eine Basis von ${o}(K,E_3)$.

Au{\ss}erdem ist 
\[
\g = \{a_1\cdot e_1 + a_2\cdot e_2 + a_3\cdot e_3 \;|\; a_1,a_2,a_3 \in K\},
\]
wobei $e_1 = \smatde{1}{0}{0}$, $e_2 = \smatde{0}{1}{0}$, $e_3 = \smatde{0}{0}{1}$, die bilden eine Basis von $\g$.

Sei $f$ eine bijektive lineare Abbildung $\g \lrxa{f} \mathfrak{o}(K,E_3)$ mit $f(e_1)=u_1$, $f(e_2)= -u_2$, $f(e_3) = u_3$, dann gilt:
\[
\ba{rcccccccl}
f([e_1, e_2]) & = & f(e_3) &=& u_3 &=& [u_1, -u_2] &=& [f(e_1),f(e_2)]\\
f([e_1, e_3]) & = & f(-e_2) &=& u_2 &=& [u_1, u_3] &=& [f(e_1),f(e_3)]\\
f([e_2, e_3]) & = & f(e_1) &=& u_1 &=& [-u_2, u_3] &=& [f(e_2),f(e_3)]
\ea
\]
%Da $\g$ und $\mathfrak{o}(K,E_3)$ Lie-Algebren sind, gelten automatisch die anderen drei entsprechenden Gleichungen.

Seien  $g,g' \in \g$, es existieren $a_i, b_i\in K, i\in[1,3]$ mit $g = a_1\cdot e_1 + a_2\cdot e_2 + a_3\cdot e_3$ und\\ $g' = b_1\cdot e_1 + b_2\cdot e_2 + b_3\cdot e_3$.
Dann gilt
\[
\barcl
f([g,g']) & = & f([a_1\cdot e_1 + a_2\cdot e_2 + a_3\cdot e_3,\;b_1\cdot e_1 + b_2\cdot e_2 + b_3\cdot e_3])\\
		  & = & \sum\limits_{i,j\in[1,3]} a_i\cdot b_j \cdot f[e_i,\,e_j]\\
		  & = & \sum\limits_{i,j\in[1,3],i<j}  (a_i\cdot b_j) \cdot f[e_i,\,e_j]+\sum\limits_{i,j\in[1,3],i>j} (a_i\cdot b_j) \cdot f[e_i,\,e_j]\\
		  & = & \sum\limits_{i,j\in[1,3],i<j}  (a_i\cdot b_j) \cdot f[e_i,\,e_j] + \sum\limits_{i,j\in[1,3],i<j} (a_j\cdot b_i) \cdot f[e_j,\,e_i]\\
		  & = & \sum\limits_{i,j\in[1,3],i<j} (a_i\cdot b_j - a_j\cdot b_i)\cdot f[e_i,\,e_j]\\
		  & = & \sum\limits_{i,j\in[1,3],i<j} (a_i\cdot b_j - a_j\cdot b_i)\cdot [f(e_i),f(e_j)]\\
		  & = & \sum\limits_{i,j\in[1,3]} a_i\cdot b_j \cdot [f(e_i),\,f(e_j)]\\
		  & = & [\sum\limits_{i\in[1,3]}a_i\cdot f(e_i) \;, \sum\limits_{j\in[1,3]}b_j\cdot f(e_j)]\\
		  & = & [f(g),f(g')]
\ea
\]
Somit ist f ein bijektive Morphismus von Lie-Algebren, daher ist $\g \simeq\mathfrak{o}(K,E_3)$.
\end{AG}
\begin{AG}
\rm
(1) Man pr\"ufe, dass $\mathfrak{der}(A)$ bez\"uglich der Lie-Klammern abgeschlossen ist.

Seien $d, d' \in\mathfrak{der}(A)\subset \mathfrak{gl}(A) $ wir haben $[d,d']= d\circ d' - d' \circ d$ sowie f\"ur $a, b \in A$
\[
\barcl
[d,d'](a\cdot b) & = & (d\circ d' - d' \circ d)(a\cdot b)\\
				 & = & d\circ d'(a\cdot b) - d'\circ d(a\cdot b)\\
				 & = & d(d'(a)\cdot b + a\cdot d'(b)) - d'(d(a)\cdot b + a\cdot d(b)\\
				 & = & d\circ d'(a)\cdot b + d'(a)\cdot d(b) + d(a)\cdot d'(b) + a\cdot d\circ d'(b)\\
				 &   & -d'\circ d(a)\cdot b - d(a)\cdot d'(b) - d'(a)\cdot d(b) -a\cdot d'\circ d(b)\\
				 & = & d\circ d'(a)\cdot b + a\cdot d\circ d'(b) - d'\circ d(a)\cdot b -a\cdot d'\circ d(b)\\
				 & = & (d\circ d'(a) - d'\circ d (a)) \cdot b + a \cdot( d\circ d'(b)-d'\circ d(b))\\
				 & = & [d,d'](a)\cdot b + a\cdot [d,d'](b)
\ea
\]
Also ist $[d,d']\in \mathfrak{der}(A)$.

(2) Sei $d\in \mathfrak{gl}(K^{2\times 2})$, dann gilt nach Definition
\[
d \in  \mathfrak{der}(K^{2\times 2}) \Leftrightarrow \forall a,b \in K^{2\times 2}: d(a\cdot b) = d(a)\cdot b + a \cdot d(b) 
\]
Seien $e_1,e_2,e_3,e_4$ eine Basis von $K^{2\times 2}$.

Seien $a = \sum\limits_{i\in [1,4] }a_i\cdot e_i$, $b = \sum\limits_{i\in [1,4] }b_i\cdot e_i$ wobei $a_i, b_i \in K , i\in[1,4]$.

Wir haben
\[
\barcl
d(a\cdot b) & = &d\big((\sum\limits_{i\in [1,4] }a_i\cdot e_i )\cdot (\sum\limits_{i\in [1,4] }b_i\cdot e_i)\big)\\
			& = & d(\sum\limits_{i,j\in[1,4]} a_i\cdot b_j \cdot e_i\cdot e_j)\\
			& = & \sum\limits_{i,j\in[1,4]}a_i\cdot b_j \cdot d(e_i\cdot e_j)
\ea
\]
und
\[
\barcl
d(a)\cdot b + a\cdot d(b) & = & d(\sum\limits_{i\in [1,4] }a_i\cdot e_i)\cdot (\sum\limits_{i\in [1,4] }b_i\cdot e_i) + \sum\limits_{i\in [1,4] }a_i\cdot e_i \cdot d(\sum\limits_{i\in [1,4] }b_i\cdot e_i)\\
						  & = & \sum\limits_{i,j\in [1,4]} a_i\cdot b_j \cdot (d(e_i)\cdot e_j +e_i\cdot d(e_j))
\ea
\]
Also gilt $d\in \mathfrak{der}(K^{2\times 2})  \Leftrightarrow \forall i,j \in[1,4] :d(e_i\cdot e_j) = d(e_i)\cdot e_j +e_i\cdot d(e_j)$

W\"ahle nun $e_1 = \smatzz{1}{0}{0}{0},\, e_2 = \smatzz{0}{1}{0}{0},\,e_3 = \smatzz{0}{0}{1}{0},\,e_4 = E_2$ als Basis.

Aus $d(E_2\cdot E_2) = d(E_2)\cdot E_2 + E_2\cdot d(E_2)$ folgt $d(E_2) = 0$. Dann gilt immer
\[
d(a\cdot b) = d(a)\cdot b + a\cdot d(b) 
\]
falls $a=E_2$ oder $b=E_2$.

Nach Berechnung ergibt sich
\[
\ba{lclcl}
e_1\cdot e_1 = e_1 & ,& e_1\cdot e_2 = e_2 & ,&e_1\cdot e_3 = 0\\
e_2 \cdot e_1 = 0  & , & e_2\cdot e_2 = 0   &, &e_2\cdot e_3 = e_1\\
e_3\cdot e_1 = e_3 & , & e_3\cdot e_2 = E_2-e_1 & ,& e_3\cdot e_3 = 0 
\ea
\]  
Sei nun $g\in \mathfrak{gl}(K^{2\times 2})$ mit
\[
\barcl
K^2 	& \lrxa{g} &K^2\\
   E_2  & \mapsto  & 0\\
   e_1  & \mapsto  & a_1\cdot e_1 + a_2 \cdot e_2 + a_3 \cdot e_3 + a_4\cdot E_2\\
   e_2  & \mapsto  & b_1\cdot e_1 + b_2 \cdot e_2 + b_3 \cdot e_3 + b_4\cdot E_2\\
   e_3  & \mapsto  & c_1\cdot e_1 + c_2 \cdot e_2 + c_3 \cdot e_3 + c_4\cdot E_2\\
\ea
\]
Wir haben neun Gleichungen $ g(e_i \cdot e_j) = g(e_i)\cdot e_j + e_i\cdot g(e_j);\; i,j\in[1,3]$, um alle diese neun Gleichungen zu erf\"ullen, muss gelten
\[
\barcl
g(\smatzz{1}{0}{0}{0}) & = & \smatzz{0}{a}{b}{0}\\
g(\smatzz{0}{1}{0}{0}) & = & \smatzz{-b}{c}{0}{b}\\
g(\smatzz{0}{0}{1}{0}) & = & \smatzz{-a}{0}{-c}{a}\\
g(\smatzz{1}{0}{0}{1}) & = & \smatzz{0}{0}{0}{0}\\
g(\smatzz{0}{0}{0}{1}) & = & \smatzz{0}{-a}{-b}{0}
\ea
\]
wobei $a,b,c \in K$.
Umgekehrt liegt jedes $g\in  \mathfrak{gl}(K^{2\times 2})$ mit diesen Eigenschaften tats\"achlich in $\mathfrak{der}(K^{2\times 2})$.

Dann bilden $g_1,g_2,g_3$ eine Basis von $\mathfrak{der}(K^{2\times 2})$ mit
\[
\barcl
g_1 &=  &\big(\smatzz{p}{q}{r}{s} \mapsto \smatzz{-r}{p-s}{0}{r} \big)\\
g_2 & = &\big(\smatzz{p}{q}{r}{s} \mapsto \smatzz{-q}{0}{p-s}{q}\big)\\
g_3 & = &\big(\smatzz{p}{q}{r}{s} \mapsto \smatzz{0}{q}{-r}{0}\big)
\ea
\]
und
\[
[g_1,g_2] = g_1\circ g_2 - g_2 \circ g_1 = \big(\smatzz{p}{q}{r}{s} \mapsto (\smatzz{s-p}{0}{-2r}{p-s}-\smatzz{s-p}{-2q}{0}{p-s} ) \big) = \big( \smatzz{p}{q}{r}{s} \mapsto \smatzz{0}{2q}{-2r}{0}\big) = 2\cdot g_3
\]
\[
[g_1,g_3] = g_1 , [g_2,g_3] = -g_2
\]
Nun w\"ahle auch eine Basis in $\mathfrak{sl}_2(K)$, n\"amlich
\[
\{b_1=\smatzz{0}{0}{1}{0} , b_2=\smatzz{0}{1}{0}{0}, b_3=\smatzz{\frac{1}{2}}{0}{0}{-\frac{1}{2}}   \}
\]
Die Berechnung ergibt
\[
[b_1,b_2] = 2\cdot b_3 ;\; [b_1,b_3] = b_1 ;\; [b_2,b_3] = [b_1^t, b_3^t] = [b_3, b_1]^t = -b_2
\]
Sei $f$ eine lineare Abbildung mit 
\[
\ba{rclc}
\mathfrak{der}(K^{2\times 2}) &\lrxa{f}& \mathfrak{sl}_2(K)&\\
g_i &\mapsto& b_i& i\in[1,3]\\
\ea
\]
Dann ist $f$ ein Morphismus von Lie-Algebren dank der obigen Berechnungen, und es gilt $\mathfrak{der}(K^{2\times 2}) \simeq \mathfrak{sl}_2(K)$
\end{AG}
\end{document}
Aufgabe 1\\	
(1). Sei $\g=K^{3\times1}$ mit $\left[\left(\begin{array}{c} \alpha \\ \beta \\ \gamma \end{array}\right),\left(\begin{array}{c} \alpha' \\ \beta' \\ \gamma' \end{array}\right)\right]=\left(\begin{array}{c} \beta\gamma'-\gamma\beta' \\ \gamma\alpha'-\alpha\gamma' \\ \alpha\beta'-\beta\alpha' \end{array}\right)$ f\"{u}r $\left(\begin{array}{c} \alpha \\ \beta \\ \gamma \end{array}\right),\left(\begin{array}{c} \alpha' \\ \beta' \\ \gamma' \end{array}\right)\in K^{3\times1}$, 
\\Zu zeigen : $\g$ ist eine Lie-Algebra.

Die Abbildung ist bilinear.

1.F\"{u}r $g=\left(\begin{array}{c} \alpha \\ \beta \\ \gamma \end{array}\right)\in K^{3\times1}$, $[g,g]=\left[\left(\begin{array}{c} \alpha \\ \beta \\ \gamma \end{array}\right),\left(\begin{array}{c} \alpha \\ \beta \\ \gamma \end{array}\right)\right]=\left(\begin{array}{c} \beta\gamma-\gamma\beta \\ \gamma\alpha-\alpha\gamma \\ \alpha\beta-\beta\alpha \end{array}\right)=0$\\
2.Seien $g,g',g''\in K^{3\times1}$ mit $g=\left(\begin{array}{c} \alpha \\ \beta \\ \gamma \end{array}\right),g'=\left(\begin{array}{c} \alpha' \\ \beta' \\ \gamma' \end{array}\right),g''=\left(\begin{array}{c} \alpha'' \\ \beta'' \\ \gamma'' \end{array}\right)$\\
Dann gilt 
\[[g,[g',g'']]+[g',[g'',g']]+[g'',[g,g']]=\left(\begin{array}{c} \beta\alpha'\beta''-\beta\beta'\alpha''-\gamma\gamma'\alpha''+\gamma\alpha'\gamma'' \\ \gamma\beta'\gamma''-\gamma\gamma'\beta''-\alpha\alpha'\beta''+\alpha\beta'\alpha'' \\ \alpha\gamma'\alpha''-\alpha\alpha'\gamma''-\beta\beta'\gamma''+\beta\gamma'\beta'' \end{array}\right)+\]\[\left(\begin{array}{c} \beta'\alpha''\beta-\beta'\beta''\alpha-\gamma'\gamma''\alpha+\gamma'\alpha''\gamma \\ \gamma'\beta''\gamma-\gamma'\gamma''\beta-\alpha'\alpha''\beta+\alpha'\beta''\alpha \\ \alpha'\gamma''\alpha-\alpha'\alpha''\gamma-\beta'\beta''\gamma+\beta'\gamma''\beta \end{array}\right)+\left(\begin{array}{c} \beta''\alpha\beta'-\beta''\beta\alpha'-\gamma''\gamma\alpha'+\gamma''\alpha\gamma' \\ \gamma''\beta\gamma'-\gamma''\gamma\beta'-\alpha''\alpha\beta'+\alpha''\beta\alpha' \\ \alpha''\gamma\alpha'-\alpha''\alpha\gamma'-\beta''\beta\gamma'+\beta''\gamma\beta' \end{array}\right)=0\]
Dann ist $\g$ eine Lie-Algebra.\\
\\
(2). Sei $n\geq0, a\in K^{n\times n}$, zu zeigen: $\mathfrak{o}(K,a):=\{g\in\mathfrak{gl}_n(K):g^ta=-ag\}\leq\mathfrak{gl}_n(k)$\\
Es ist zu zeigen: Seien $g,\bar{g}\in\mathfrak{o}(K,a)$, $[g,\bar{g}]\in\mathfrak{o}(K,a)\Leftrightarrow[g,\bar{g}]^ta=-a[g,\bar{g}]$\\
\[[g,\bar{g}]^ta=(g\bar{g}-\bar{g}g)^ta=(g\bar{g})^ta-(\bar{g}g)^ta=\bar{g}^tg^ta-g^t\bar{g}^ta\]
Da $g,\bar{g}\in\mathfrak{o}(K,a)\Rightarrow g^ta=-ag,\bar{g}^ta=-a\bar{g}$\\
\[\bar{g}^tg^ta-g^t\bar{g}^ta=-\bar{g}^tag+g^ta\bar{g}=a\bar{g}g-ag\bar{g}=-a(g\bar{g}-\bar{g}g)=-a[g,\bar{g}]\]
$[g,\bar{g}]^ta=-a[g,\bar{g}]$ also $[g,\bar{g}]\in\mathfrak{o}(K,a)$, dann ist $\mathfrak{o}(K,a)\leq\mathfrak{gl}_n(k)$\\
\\
(3). Sei $f:\mathfrak{h}\rightarrow\mathfrak{k}$ ein Isomorphismus von Liealgebren, zu zeigen: $f^{-1}:\mathfrak{k}\rightarrow\mathfrak{h}$ ist auch ein Isomorphismus.\\
Es ist zu zeigen : f\"{u}r $k,k'\in\mathfrak{k}, f^{-1}([k,k'])=[f^{-1}(k),f^{-1}(k')]$\\
Da f ist ein Isomorphismus, $[f(f^{-1}(k)),f(f^{-1}(k'))]=f([f^{-1}(k),f^{-1}(k')]$
\[\Rightarrow f^{-1}([k,k'])=f^{-1}([f(f^{-1}(k)),f(f^{-1}(k'))])=f^{-1}\circ f([f^{-1}(k),f^{-1}(k')])=[f^{-1}(k),f^{-1}(k')]\]
Dann ist $f^{-1}$ auch ein Isomorphismus.

(4).Wir wissen 
\[
\mathfrak{o}(K,E_3) = \{a_1\cdot u_1 + a_2\cdot u_2 + a_3\cdot u_3 \;|\; a_1,a_2,a_3 \in K\},
\]
wobei $u_1 = \smatdd{0}{1}{0}{-1}{0}{0}{0}{0}{0}$, $u_2=\smatdd{0}{0}{1}{0}{0}{0}{-1}{0}{0}, u_3 = \smatdd{0}{0}{0}{0}{0}{1}{0}{-1}{0}$, die bilden eine Basis von ${o}(K,E_3)$.

Au{\ss}erdem ist 
\[
\g = \{a_1\cdot e_1 + a_2\cdot e_2 + a_3\cdot e_3 \;|\; a_1,a_2,a_3 \in K\},
\]
wobei $e_1 = \smatde{1}{0}{0}$, $e_2 = \smatde{0}{1}{0}$, $e_3 = \smatde{0}{0}{1}$, die bilden eine Basis von $\g$.

Sei $f$ eine bijektive lineare Abbildung $\g \lrxa{f} \mathfrak{o}(K,E_3)$ mit $f(e_1)=u_1$, $f(e_2)= -u_2$, $f(e_3) = u_3$, dann gilt:
\[
\ba{rcccccccl}
f([e_1, e_2]) & = & f(e_3) &=& u_3 &=& [u_1, -u_2] &=& [f(e_1),f(e_2)]\\
f([e_1, e_3]) & = & f(-e_2) &=& u_2 &=& [u_1, u_3] &=& [f(e_1),f(e_3)]\\
f([e_2, e_3]) & = & f(e_1) &=& u_1 &=& [-u_2, u_3] &=& [f(e_2),f(e_3)]
\ea
\]
%Da $\g$ und $\mathfrak{o}(K,E_3)$ Lie-Algebren sind, gelten automatisch die anderen drei entsprechenden Gleichungen.

Seien  $g,g' \in \g$, es existieren $a_i, b_i\in K, i\in[1,3]$ mit $g = a_1\cdot e_1 + a_2\cdot e_2 + a_3\cdot e_3$ und\\ $g' = b_1\cdot e_1 + b_2\cdot e_2 + b_3\cdot e_3$.
Dann gilt
\[
\barcl
f([g,g']) & = & f([a_1\cdot e_1 + a_2\cdot e_2 + a_3\cdot e_3,\;b_1\cdot e_1 + b_2\cdot e_2 + b_3\cdot e_3])\\
		  & = & \sum\limits_{i,j\in[1,3]} a_i\cdot b_j \cdot f[e_i,\,e_j]\\
		  & = & \sum\limits_{i,j\in[1,3],i<j}  (a_i\cdot b_j) \cdot f[e_i,\,e_j]+\sum\limits_{i,j\in[1,3],i>j} (a_i\cdot b_j) \cdot f[e_i,\,e_j]\\
		  & = & \sum\limits_{i,j\in[1,3],i<j}  (a_i\cdot b_j) \cdot f[e_i,\,e_j] + \sum\limits_{i,j\in[1,3],i<j} (a_j\cdot b_i) \cdot f[e_j,\,e_i]\\
		  & = & \sum\limits_{i,j\in[1,3],i<j} (a_i\cdot b_j - a_j\cdot b_i)\cdot f[e_i,\,e_j]\\
		  & = & \sum\limits_{i,j\in[1,3],i<j} (a_i\cdot b_j - a_j\cdot b_i)\cdot [f(e_i),f(e_j)]\\
		  & = & \sum\limits_{i,j\in[1,3]} a_i\cdot b_j \cdot [f(e_i),\,f(e_j)]\\
		  & = & [\sum\limits_{i\in[1,3]}a_i\cdot f(e_i) \;, \sum\limits_{j\in[1,3]}b_j\cdot f(e_j)]\\
		  & = & [f(g),f(g')]
\ea
\]
Somit ist f ein bijektive Morphismus bzw. Isomorphismus von Lie-Algebren, daher ist \linebreak $\g \lrxa{\sim}\mathfrak{o}(K,E_3)$.
\end{document}
